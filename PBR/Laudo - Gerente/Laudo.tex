\documentclass[12pt,a4paper,final]{article}
\usepackage[utf8]{inputenc}
\usepackage{amsmath}
\usepackage{amsfonts}
\usepackage{amssymb}
\usepackage{graphicx}
\usepackage{enumitem}
\usepackage{amsmath}
\usepackage[portuguese]{babel}
\usepackage[utf8]{inputenc}
\usepackage{pdfpages}
\usepackage[portuguese]{babel}
\usepackage[T1]{fontenc}
\usepackage{amsmath}
\usepackage{amsfonts}
\usepackage{amssymb}
\usepackage{graphicx}
\usepackage[left=2.5cm,right=2.5cm,top=2.5cm,bottom=2.5cm]{geometry}
\begin{document}

      	\begin{titlepage}
       	\LARGE
       	\begin{center}
       	\vspace{5cm} 
       	\textbf{Universidade Tecnológica Federal do Paraná \\ \vspace{1.8cm}}
       	\includegraphics[scale=0.35]{logoutfpr.jpg} \\ \vspace{1.8cm}
       	\textit{Engenharia de Software II} \vspace{2cm} \\
       	Luan Bodner do Rosário \\ 1509950 \vspace{2cm} \\ 
       	PROJETO 1 - LAUDO DO GERENTE PBR \vspace{2cm} \\
      	Departamento de Ciência da Computação (DACOM) 
       	
   	\end{center}
    \end{titlepage}	

\tableofcontents
\newpage
\section{Análise do Documento}
\begin{enumerate}
\item Dados pessoais dos funcionários não estabelecidos: Os dados pessoais dos trabalhadores do HemoSystem não foram definidos explicitamente na documentação, criando problemas de interpretação entre os analisadores.

\item Interface entre os funcionários do sistema não está estabelecida: Dados referentes à coleta de sangue entre as fases de análise e exames não estão coerentes.

\item Interface dentro do próprio sistema não está coerente: Informações dadas durante parte do processo de coleta de dados não é necessária ou não está presente quando necessário enquanto o doador está no processo de cadastramento

\item Nível de segurança do sistema não está devidamente definido: A diferença entre funcionário e gerente não está bem estabelecida, ou seja, privilégios do gerente não são definidos e as restrições dos funcionários também estão vagas.

\item Difícil interpretação devido a falta de detalhes da especificação: Como dito anteriormente, como alguns atributos não foram pré-definidos, os analistas do PBR não tiveram diferentes interpretações do mesmo problema.

\item Faltam detalhes da documentação : Além do caso do funcionário, foi difícil definir alguns domínios quanto ao atributo de aptidão física.

\item Tipos de dados não definidos explicitamente em alguns casos : Tipos das variáveis de atributo e qual seria o escopo de cada uma delas foram mal definidas.
\end{enumerate}
\newpage
\section{Horas de Trabalho}
\begin{itemize}
\item Projetista:
\begin{itemize}
\item 22/03/2016: 15:00 - 15:34;
\item 30/03/2016: 16:15 - 16:55.
\end{itemize}
A estimativa foi de 12  horas, entretanto ao que foi visto houve superestimação da carga horária.

\item Testador: 
\begin{itemize}
\item 31/03/2016 15:30 - 17:00.
\end{itemize}
A estimativa foi de 8 horas, mas o tempo gasto foi muito menor do que o esperado.  

\item Usuário:
\begin{itemize}
\item 02/04/2016: 13:00 - 13:30;
\item 03/04/2016: 12:30 - 16:00.
\end{itemize}
\end{itemize}
A estimativa foi de 3.5 horas, mas o tempo gasto foi maior do que o esperado. 

\end{document}