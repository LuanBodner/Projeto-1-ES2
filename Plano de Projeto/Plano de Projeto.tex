\documentclass[12pt,a4paper,final]{report}
\usepackage[utf8]{inputenc}
\usepackage{amsmath}
\usepackage{amsfonts}
\usepackage{amssymb}
\usepackage{graphicx}
\usepackage{enumitem}
\usepackage[portuguese]{babel}
\author{Gerente: Luan Bodner do Rosário}
\title{Plano de Projeto - Projeto 1}
\begin{document}
\maketitle
\section*{Planejamento}

Este documento tem o intuíto de definir e dividir as principais tarefas do sistema a ser implementado para a disciplina de Engenharia de Software 2.\\
Para a definição das tarefas será utilizado o método EAP(Estrutura Analítica do Projeto) e com base nessa divisão, será utilizado o PERT(\textit{Program Evaluation and Review Technique}) para fazer a estimativa de tempo de cada uma dessas partes do sistema.
A EAP do projeto é definida conforme:

\begin{enumerate}
\item \textbf{Hemosystem}
\begin{enumerate}[label*=\arabic*.]
\item \textbf{Documentação}
\begin{enumerate}[label*=\arabic*.]
\item \textit{Perspective Base Reading}
\item Atualização do Documento de Requisitos
\item Diagramas de Caso de Uso
\item Diagramas de Classe
\item Revisão dos Documentos
\end{enumerate}
\item \textbf{Infraestrutura}
\begin{enumerate}[label*=\arabic*.]
\item Definição do BD
\item Criação do \textit{Script}
\item Revisão do BD
\end{enumerate}
\item \textbf{Desenvolvimento}
\begin{enumerate}[label*=\arabic*.]
\item Criação das classes base
\item Implementação das funcionalidades definidas pelo usuário
\item Implementação da Interface entre as classes
\item Definição das \textit{views} do BD
\end{enumerate}
\item \textbf{Interface}
\begin{enumerate}[label*=\arabic*.]
\item Definição das telas
\item Conectar a interface com as funcionalidades
\end{enumerate}
\item \textbf{Testes}

\end{enumerate}
\end{enumerate}


\newpage
As tarefas a serem implementadas prévia às tarefas de programação são:

\begin{enumerate}
\item Documentação do Projeto
\begin{enumerate}

\item Perspective Base Reading : Tarefa já concluida e documentada no laudo presente no repositório do Projeto.

\item Atualização do Documento de Requisitos: Tarefa já concluida e documentada no laudo presente no repositório do Projeto.

\item Diagrama de Classe : Tarefa que ainda deverá ser realizada pelo Projetista Josimar Loch para que possa ser levado para revisão e refinamento.
\begin{itemize}
\item Estimativa de tempo para término da tarefa : \\
Dada pela Fórmula : TE = O + 4 M + P / 6\\
Tempo Otimista (O) = 2.5h\\
Tempo Mais Provável (M) = 5h\\
Tempo Pessimista (P) = 10h\\
Tempo Esperado Total = 7h
\end{itemize} 

\item Diagrama de Caso de Uso : Também sob a responsabilidade do Projetista do projeto.
\begin{itemize}
\item Estimativa de tempo para término da tarefa : 
Tempo Otimista (O) = 2h\\
Tempo Mais Provável (M) = 4h\\
Tempo Pessimista (P) = 6h\\
Tempo Esperado Total = 5h
\end{itemize} 

\item Controle de Qualidade : Após a realização da primeira fase da criação de diagramas, ou mesmo concorrentemente com a sua realização, os diagramas devem ser analisados e refinados para que o processo de programação ocorra o mais rápido possível. Controle de qualidade será uma tarefa conjunta entre todos os membros da equipe.
\begin{itemize}
\item Estimativa de tempo para término da tarefa :\\
Tempo Otimista (O) = 1h\\
Tempo Mais Provável (M) = 2h\\
Tempo Pessimista (P) = 3h\\
Tempo Esperado Total = 2.5h
\end{itemize} 

\item Revisão : Após a revisão ser concluída e os erros listados previamente na análise PBR e encontrados na documentação forem encontrados, a documentação deverá passar por mudanças e submetidos em forma final. Responsabilidade do Projetista Josimar.
\begin{itemize}
\item Estimativa de tempo para término da tarefa :
Tempo Otimista (O) = 0.5h\\
Tempo Mais Provável (M) = 1h\\
Tempo Pessimista (P) = 3h\\
Tempo Esperado Total = 1.25h
\end{itemize} 
\end{enumerate}

Esses diagramas devem estar em um formato acessível para que o Programador Felipe Veiga Ramos possa lê-los com mais facilidade.\\
Todas essas sub-etapas podem (e devem) ser feitas concorrentemente.\\
As ferramentas utilizadas nessa parte do desenvolvimento são:
\begin{enumerate}
\item Astah
\item Github
\item Issue Tracker/Github
\end{enumerate}

\item Infraestrutura
\begin{enumerate}
\item Criação do Banco de Dados : Após a realização dos diagramas, o banco de dados deve ser criado com base nos dados necessários. Isso será uma tarefa conjunta entre o Projetista e o Programador.
Tempo Otimista (O) = 4h\\
Tempo Mais Provável (M) = 5h\\
Tempo Pessimista (P) = 8h\\
Tempo Esperado Total = 8h

\item Criação do \textit{Script} : Após os dados a serem armazenados forem definidos, deve-se criar o \textit{script} do banco de dados. Tarefa feita entre Projetista e Programador.
Tempo Otimista (O) = 2h\\
Tempo Mais Provável (M) = 3.5h\\
Tempo Pessimista (P) = 7h\\
Tempo Esperado Total = 4.6h

\item Revisão BD : O banco deve ser revisado pelo Gerente do Projeto para evitar erros prévios ao início de desenvolvimento.
Tempo Otimista (O) = 1h\\
Tempo Mais Provável (M) = 2h\\
Tempo Pessimista (P) = 4h\\
Tempo Esperado Total = 2.5h
\end{enumerate}

As ferramentas utilizadas nessa parte do projeto está sujeito a escolha do Programador e Projetista.

\item Desenvolvimento
\begin{enumerate}

\item Criação das Classes Base : Criação das estruturas básicas e classes básicas para as operações lógicas do sistema a ser implementado. Todas as tarefas do desenvolvimento é tarega do Programador.
Tempo Otimista (O) = 3h\\
Tempo Mais Provável (M) = 5h\\
Tempo Pessimista (P) = 8h\\
Tempo Esperado Total = 7.5h

\item Implementação das funcionalidades : Com base nos diagramas e casos de uso definidos no documento original do projeto e o documento revisado com informações novas, as funcionalidades devem ser implementadas.
Tempo Otimista (O) = 12h\\
Tempo Mais Provável (M) = 15h\\
Tempo Pessimista (P) = 20h\\
Tempo Esperado Total = 40h

\item Implementação da interface entre as classes : Após as partes singulares do sistema estiverem implementadas, as partes do sistema devem ser conectadas de acordo com as especificações feitas.
Tempo Otimista (O) = 5h\\
Tempo Mais Provável (M) = 6h\\
Tempo Pessimista (P) = 8h\\
Tempo Esperado Total = 10h

\item Definição das \textit{views} do BD : Por questões de segurança, deve-se criar views diferentes no banco para cada nível de prioridade dos usuários.
Tempo Otimista (O) = 4h\\
Tempo Mais Provável (M) = 6h\\
Tempo Pessimista (P) = 8h\\
Tempo Esperado Total = 9h
\end{enumerate}

\item Interface
\begin{enumerate}
\item Definição das telas : Após o programa estiver completado, a interface com o usuário deve ser definida.
Tempo Otimista (O) = 3h\\
Tempo Mais Provável (M) = 4h\\
Tempo Pessimista (P) = 5h\\
Tempo Esperado Total = 5.5h
\item Conectar a interface com as funcionalidades : Após a interface estiver "desenhada", os módulos resultantes devem ser conectados ao código do programa e suas entradas/saídas.
Tempo Otimista (O) = 4h\\
Tempo Mais Provável (M) = 6h\\
Tempo Pessimista (P) = 8h\\
Tempo Esperado Total = 8h
\end{enumerate}

\item Testes : Com o programa completo, ou mesmo durante o seu desenvolvimento, o Testador Paulo Batista deve fazer os testes básicos definidos no PBR para verificar se o programa está funcionando corretamente.
Tempo Otimista (O) = 4h\\
Tempo Mais Provável (M) = 6h\\
Tempo Pessimista (P) = 8h\\
Tempo Esperado Total = 8h

\end{enumerate}

\section*{Bibliografia}
http://www.workbreakdownstructure.com/\\
http://projetoseti.com.br/criar-a-estrutura-analitica-do-projeto-eap/\\
http://stakeholdernews.com.br/artigo/estimativas-tempo-e-custo-investimentos/

\section*{Horários}
Início : 6:30 da manhã, 24/05\\
Término : 11:00 da manhã, 24/05

\end{document}