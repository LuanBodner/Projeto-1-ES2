\documentclass[12pt,a4paper,final]{report}
\usepackage[utf8]{inputenc}
\usepackage{amsmath}
\usepackage{amsfonts}
\usepackage{amssymb}
\usepackage{graphicx}
\usepackage[portuguese]{babel}
\author{Gerente: Luan Bodner do Rosário}
\title{Laudo - Projeto 1}
\begin{document}
\maketitle
\section*{Análise do Documento}
\begin{itemize}
\item Dados pessoais dos funcionários não estabelecidos: Os dados pessoais dos trabalhadores do HemoSystem não foram definidos explicitamente na documentação, criando problemas de interpretação entre os analisadores.

\item Interface entre os funcionários do sistema não está estabelecida: Dados referentes à coleta de sangue entre as fases de análise e exames não estão coerentes.

\item Interface dentro do próprio sistema não está coerente: Informações dadas durante parte do processo de coleta de dados não é necessária ou não está presente quando necessário enquanto o doador está no processo de cadastramento

\item Nível de segurança do sistema não está devidamente definido: A diferença entre funcionário e gerente não está bem estabelecida, ou seja, privilégios do gerente não são definidos e as restrições dos funcionários também estão vagas.

\item Difícil interpretação devido a falta de detalhes da especificação: Como dito anteriormente, como alguns atributos não foram pré-definidos, os analistas do PBR não tiveram diferentes interpretações do mesmo problema.

\item Faltam detalhes da documentação : Além do caso do funcionário, foi difícil definir alguns domínios quanto ao atributo de aptidão física.

\item Tipos de dados não definidos explicitamente em alguns casos : Tipos das variáveis de atributo e qual seria o escopo de cada uma delas foram mal definidas.

\end{itemize}

\end{document}